% !TeX root=../../../main.tex

\chapter{روش پیشنهادی}
%\thispagestyle{empty} 
\section{مقدمه} 
پس از آشنایی با روش‌های پیشین که برای حل مسئله مشابه مورد استفاده قرار گرفته‌اند، حال می‌توانیم به معرفی و تشریح روش‌های پیشنهادی خود برای حل مسئله پیش رو بپردازیم. در این فصل ابتدا داده‌های ورودی مسئله را همراه با فرضیات در نظر گرفته شده بیان می‌کنیم و پس از آن دو روش پیشنهادی متفاوت را بیان خواهیم نمود. در روش اول که به رویکردهای پیشین نزدیک‌تر است با تغییری از جنس روش‌‌های نوین در مراحل میانی به یک روش جدید می‌رسیم که به علت افزایش سرعت همگرایی می‌توان فرض و داده‌های جدیدی را از طریق \lr{CNV} به آن افزود و پاسخ گرفت. اما روش دوم کاملا متفاوت بوده و با رویکردی جدید در حوزه یادگیری ماشین همراه است که به کمک یادگیری تقویتی به حل مسئله مورد نظر می‌پردازد.


\section{معرفی دادگان ورودی}
قبل از وارد شدن به بخش روش‌های پیشنهادی نیاز است تا دادگان ورودی را مشخص و معرفی نماییم تا در قسمت‌های بعدی بتوانیم از نماد‌های معرفی شده در این بخش استفاده نماییم.
\\
دادگان مورد استفاده 

\section{روش پیشنهادی اول (درخت‌بازی)}

\subsection{پیش‌پردازش}
قبل از شروع باید بر روی داده‌ها یک پیش‌پردازش اعمال کنیم که وابسته به سیاست درنظر گفته شده می‌تواند باعث تغییر در پاسخ نهایی نیز شود. به این منظور داده‌هایی که \lr{miss} شده‌اند با روش‌های زیر ‌می‌تواند برای ورود به مرحله بعد تخمین زده شود.
\subsubsection{تصادفی}
پر کردن کاملا تصادفی میس‌ها



\newpage
	\begin{table}[ht]
		\caption{اندیس‌های به کار رفته در مدل ریاضی}
		\label{tab:modelIndices}
		\centering
		\onehalfspacing
		\begin{tabularx}{0.9\textwidth}{|r|X|}
			\hline
			$I, J$	& بیماران \\
			\hline
			$k$		& مرحله زمان‌بندی (بستری، اتاق عمل، ریکاوری) \\
			\hline
			$L_k$	& ماشین (تخت یا اتاق عمل) در مرحله $k$ \\
			\hline
			$n$		&  جراح \\
			\hline
		\end{tabularx}
	\end{table}
	
	\begin{table}[ht]
		\caption{پارامترهای مدل ریاضی}
		\label{tab:modelParameters}
		\centering
		\onehalfspacing
		\begin{tabularx}{0.9\textwidth}{|r|X|}
			\hline
			$t_{ik}$			& زمان خدمت‌دهی به بیمار در مرحله $k$ام \\
			\hline
			$\tilde{t}_{ik}$	& زمان فاری خدمت‌دهی به بیمار در محله $k$ام \\
			\hline
			$t_{ik}^p$			& مقدار بدبینانه (حداکثر) برای زمان خدمت‌دهی به بیمار در مرحله $k$ام \\
			\hline
			$t_{ik}^m$			& محتمل‌ترین مقدار برای زمان خدمت‌دهی به بیمار در مرحله $k$ام \\
			\hline
			$t_{ik}^o$			& مقدار خوشبینانه (حداقل) برای زمان خدمت‌دهی به بیمار در مرحله $k$ام \\
			\hline
		\end{tabularx}
	\end{table}
	
	\begin{table}[ht]
		\caption{متغیرهای مدل ریاضی}
		\label{tab:modelVariables}
		\centering
		\onehalfspacing
		\begin{tabularx}{0.9\textwidth}{|r|X|}
			\hline
			$X_{ild_{k}}$	& متغیر صفر-یک تخصیص بیمار به تخت/اتاق عمل\\
			\hline
			$S_{ild_{k}}$	& زمان شروع خدمت‌دهی به بیمار \\
			\hline
			$Y_{ijkl_{k}}$	& متغیر صفر-یک توالی بیماران \\
			\hline
			$V_{ni}$		& متغیر صفر-یک تخصیص جراح به بیمار‍‍ \\
			\hline
		\end{tabularx}
	\end{table}
	
