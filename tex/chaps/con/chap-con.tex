% !TeX root=../../../main.tex

\chapter{بحث و نتیجه‌گیری}
% دستور زیر باعث عدم‌نمایش شماره صفحه در اولین صفحهٔ این فصل می‌شود.
%\thispagestyle{empty}


در این مقاله الگوریتم \lr{scarlet} معرفی شد که در آن به طور همزمان از \gls{snvv} (\lr{SNV}) و جهش‌های \gls{cnv} (\lr{CNA}) از داده‌های \gls{scs} برای استنباط فیلوژنی تومور استفاده شد. این الگوریتم، یک مدل تکاملی بر اساس در نظر گرفتن خطای ناشی از حذف جهش است که حذف جهش‌ها را محدود به مکان‌هایی می‌کند که شواهدی از حذف جهش‌های \gls{cnv} موجود باشد. مدلهای فیلوژنی \gls{losssupported}،  با استفاده از اطلاعات جهش‌های \gls{cnv} که به آسانی در داده‌های \gls{snvv} موجود است، نسب به مدل های \gls{dollo} یا فرض \gls{infinitesites}، \gls{conflict} کمتری در استنباط درخت فیلوژنی دارند. اگر چه به صورت طبیعی در داده‌های \gls{snvv} یک عدم قطعیت ذاتی در حضور یا عدم حضور جهش در سلول‌ها وجود دارد، اما کاهش میزان ابهام در استنباط فیلوژنی تومور منجر به افزایش \gls{accuracy} فیلوژنی استنباط شده است. در این مقاله نشان داده شد که فیلوژنی توموری استنباط شده برای بیماران مبتلا به سرطان روده از دقت و تکرارپذیری بیشتری برخوردار است و این الگوریتم در نهایت فیلوژنی‌هایی را استنباط کرد که در آن 3 حذف جهش رخ داده بود. البته این الگوریتم محدودیت‌های خاص خود را دارد. به عنوان مثال، این نوع پیاده‌سازی از الگوریتم اسکارلت مستلزم درخت \gls{cnv} به عنوان ورودی و میزان \gls{likelihood} هر یک از این درختان است. این رویکرد در مواقعی که تعداد مشخصی از تغییرات تعداد کپی وجود دارد قابل اجراست اما هنگامی که داده‌های \gls{scs} در مقیاس بزرگ انجام شود، به درختان زیادی از جهش‌های \gls{cnv} نیاز خواهد بود. 

\section{گام‌های آتی}
در ادامه برای تکمیل روش پیشنهادی در دو قسمت نیاز به بهبود وجود دارد.\\
قسمت اول مربوط به درخت اولیه است و قسمت دیگر مربوط به سرعت \lr{MCMC} می‌باشد.
\subsection{بهبود در ساخت درخت اولیه}
در حال حاضر ما درخت اولیه را به صورت تصادفی انتخاب می‌کنیم که می‌توان در این مرحله درخت اولیه را با استفاده از مفروضات مدل مکان‌های بی‌نهایت و با توجه به ماتریس ورودی بهبود بخشید. این کار باعث می‌شود تا شروع الگوریتم از نقطه بهتری باشد که در این صورت هم گام‌های لازم برای رسیدن به درخت بهینه می‌تواند کمتر شود و هم اینکه احتمال قرار گرفتن در نقاط اکسترمم نسبی را کاهش می‌دهیم.

\subsection{افزایش سرعت همگرایی \lr{MCMC}}
در این بخش نیاز است تا در دو قسمت روش پیشنهادی بهبود یابد.
\subsubsection{تنوع در گام‌ها با استراتژی معقول}
برای این بخش کاری که باید انجام شود این است که بتوان برای افزایش سرعت همگرایی از روش‌های مختلف در گام‌ها استفاده کرد. برای مثال در حال حاضر می‌توان از سه روش مختلف در هر گام استفاده نمود. روش اول تعویض دو نود در درخت می‌باشد. روش دوم جدایی یک زیر درخت و اتصال آن به محلی دیگر می‌باشد و  در نهایت روش سوم تعویض دو زیر درخت با یکدیگر می‌باشد. با انتخاب یک استراتژی مناسب بین هرکدام از این روش‌ها در گام‌های مختلف احتمالا بتوان سرعت همگرایی را افزایش داد.

\subsubsection{قرار دادن احتمال وزن‌دار به ازای هر انتخاب}
در حال حاضر ما در هرکدام از روش‌های مختلف که در بخش قبل برای گام‌های \lr{MCMC} بیان کردیم، انتخاب نودها را به صورت کاملا یکنواخت انجام می‌دهیم. در صورتی که احتمالا بتوان با تعریف فرمولی مناسب این احتمال انتخاب بین نودهای مختلف در درخت را از حالت یکنواخت خارج کرد و در نتیجه مجددا سرعت همگرایی الگوریتم را افزایش داد.
