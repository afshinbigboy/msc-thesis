% !TeX root=../../../main.tex

\chapter{روش‌های پیشین}
% دستور زیر باعث عدم‌نمایش شماره صفحه در اولین صفحهٔ این فصل می‌شود.
%\thispagestyle{empty}

\section{مقدمه}
در فصل گذشته به معرفی مفاهیم و موضوعات مرتبط با این حوزه پرداخته شد. در ادامه در این فصل با توجه به اطلاعاتی که کسب کرده‌اید به معرفی و بررسی روش‌هایی که مرتبط با موضوع این پایان‌نامه است پرداخته خواهد شد و نتایج آن‌ها را برای فرض‌های و داده‌های ورودی خود مشاهده خواهیم نمود. در این بین تا جایی که ممکن باشد به بررسی نقاط قوت و ضعف آن‌ها نیز خواهیم پرداخت و در انتهای این فصل یک جدول مقایسه بین روش‌هایی که تا به حال معرفی شده‌اند را ارائه خواهیم داد.

%در ابتدای این فصل به معرفی مقاله و روش SCITE خواهیم پرداخت که البته یکی از مقالات اصلی پایان‌نامه جاری می‌باشد و یکی از روش‌های پیشنهادی در فصل آینده نیز بر پایه همین روش می‌باشد.


\section{\gls{tumorevolutionarytreeinference} با استفاده از داده‌های \gls{scs}}
\gls{cancer} نامی است که به مجموعه¬ای از بیماری¬ها اطلاق می¬شود که از تکثیر مهار نشده سلول¬ها پدید می¬آیند. تحقیقات انجام شده نشان می¬دهد که سرطان در واقع یک فرآیند تکاملی از جهش¬های  ژنتیکی، شامل حذف و تغییر تعداد کپی ، حذف و تغییر تک نوکلئوتیدد¬ها  ، بازسازی و جایگزینی ژن¬ها در سلول¬های توموری است. در واقع تومور زمانی ایجاد می¬شود که یک سلول جهش یافته بتواند با عبور از سیستم دفاعی بدن زندگی کرده و تکثیر شود به گونه¬ای که نسبت مرگ به تولید آن گونه ایجاد شده بسیار کوچک تر از 1   (α<<1) باشد. با پیشرفت تومور، ناهنجاری¬های ژنتیکی  مختلف منجر به افزایش گروه¬های جمعیتی ناهمگنی به نام کلون  می¬شود. فرآیند تکاملی همه این کلون¬ها را می¬توان با یک درخت فیلوژنی   و آنالیز فیلوژنیتیکی از چندین کلون سلولی سرطانی مدل¬سازی کرد که می¬تواند مطالعه انواع تومور را تسهیل کند. ساختار و الگو¬های درون این درخت میزان وابستگی بین گونه¬های خاص را با توجه به تعداد و فواصل بین اجداد مشترک¬شان تعیین می¬کند. درخت¬های فیلوژنی عملکرد بارزی در توصیف فرآیند توسعه تومور دارند که بهتر از دیگر الگوریتم¬های مشابه عمل می¬کنند. تحلیل توپولوژی درخت درخت پیشرفت تومور نشان می¬دهد که مسیر توسعه تومور در طول مراحل مختلف تشکیل تومور، تا حد زیادی تغییر می¬کند و البته نتایج مبتنی بر درخت بهتر از نتایج داده¬های بدست آمده از طریق روش¬های دیگر در تشخیص تومور می¬باشد.
در حال حاضر ظهور تکنولوژی¬های بر پایه DNA یک سلول منفرد ، با هدف افزایش دانش از جنبه¬های مختلف بیولوژی سرطان، شامل بررسی زیرساخت کلونال، ردیابی تکامل تومور، شناسایی زیرکلون¬های نادر و درک ریزمحیط-های سرطانی در پیشرفت تومور، به یاری محققان این حوزه آمده و بالاترین وضوح را از تاریخچه سرطان (درخت فیلوژنی) فراهم کرده است. در واقع از آنجایی که در روش¬ توالی¬یابی تک سلولی  گونه¬های مختلف از ابتدا از هم جدا  می¬شوند، از نقطه منظر از دست دادن تنوع در زیرجمعیت بافت مورد آزمایش نداریم و به همین دلیل دقت این روش نیز نسبت به روش انبوه بالاتر می¬باشد. در کنار مزایا این روش، معایبی چون، هزینه بالا، از دست دادن سلول¬ها، جهش ثانویه در هنگام کشت، از دست دادن میزان فراوانی درون تومور حقیقی و زمان¬گیر بودن فرآیند نمونه گیری اشاره کرد.

در ابتدا استفاده از روش¬های توالی¬یابی انبوه  بدلیل اینکه حجم بالایی از اطلاعات در اثر این توالی¬یابی ایجاد می¬شود، از محبوبیت بیشتری برخوردار بود  اما با پیشرفت تکنولوژی و ظهور روش¬های نوینی چون توالی¬یابی تک¬سلولی این مهم دچار تغییر شد. در روش توالی¬یابی انبوه، نمونه¬برداری بر روی تعداد بسیار زیادی سلول ( از محدوده¬ی هزار تا میلیون سلول) صورت می¬گرفت و حجم بالای داده¬ها و امکان تفکیک پایین نواحی ناهمگن، اطلاعات کافی از ساختار درون تومور و ناهمگنی¬های درون توموری بدست نمی¬داد. در مقابل، در روش توالی¬یابی تک¬سلولی، اگر¬چه میزان هزینه نمونه¬برداری افزایش قابل¬توجهی داشت و یا میزان اطلاعات از دست رفته  و نویز موجود در داده¬های توالی¬یافته بالا بود.
